%% Generated by Sphinx.
\def\sphinxdocclass{report}
\documentclass[letterpaper,10pt,english]{sphinxmanual}
\ifdefined\pdfpxdimen
   \let\sphinxpxdimen\pdfpxdimen\else\newdimen\sphinxpxdimen
\fi \sphinxpxdimen=.75bp\relax

\PassOptionsToPackage{warn}{textcomp}
\usepackage[utf8]{inputenc}
\ifdefined\DeclareUnicodeCharacter
% support both utf8 and utf8x syntaxes
  \ifdefined\DeclareUnicodeCharacterAsOptional
    \def\sphinxDUC#1{\DeclareUnicodeCharacter{"#1}}
  \else
    \let\sphinxDUC\DeclareUnicodeCharacter
  \fi
  \sphinxDUC{00A0}{\nobreakspace}
  \sphinxDUC{2500}{\sphinxunichar{2500}}
  \sphinxDUC{2502}{\sphinxunichar{2502}}
  \sphinxDUC{2514}{\sphinxunichar{2514}}
  \sphinxDUC{251C}{\sphinxunichar{251C}}
  \sphinxDUC{2572}{\textbackslash}
\fi
\usepackage{cmap}
\usepackage[T1]{fontenc}
\usepackage{amsmath,amssymb,amstext}
\usepackage{babel}



\usepackage{times}
\expandafter\ifx\csname T@LGR\endcsname\relax
\else
% LGR was declared as font encoding
  \substitutefont{LGR}{\rmdefault}{cmr}
  \substitutefont{LGR}{\sfdefault}{cmss}
  \substitutefont{LGR}{\ttdefault}{cmtt}
\fi
\expandafter\ifx\csname T@X2\endcsname\relax
  \expandafter\ifx\csname T@T2A\endcsname\relax
  \else
  % T2A was declared as font encoding
    \substitutefont{T2A}{\rmdefault}{cmr}
    \substitutefont{T2A}{\sfdefault}{cmss}
    \substitutefont{T2A}{\ttdefault}{cmtt}
  \fi
\else
% X2 was declared as font encoding
  \substitutefont{X2}{\rmdefault}{cmr}
  \substitutefont{X2}{\sfdefault}{cmss}
  \substitutefont{X2}{\ttdefault}{cmtt}
\fi


\usepackage[Bjarne]{fncychap}
\usepackage{sphinx}

\fvset{fontsize=\small}
\usepackage{geometry}


% Include hyperref last.
\usepackage{hyperref}
% Fix anchor placement for figures with captions.
\usepackage{hypcap}% it must be loaded after hyperref.
% Set up styles of URL: it should be placed after hyperref.
\urlstyle{same}

\usepackage{sphinxmessages}




\title{open source watch Documentation}
\date{Nov 27, 2020}
\release{1.0.0}
\author{jj}
\newcommand{\sphinxlogo}{\vbox{}}
\renewcommand{\releasename}{Release}
\makeindex
\begin{document}

\pagestyle{empty}
\sphinxmaketitle
\pagestyle{plain}
\sphinxtableofcontents
\pagestyle{normal}
\phantomsection\label{\detokenize{index::doc}}
\noindent\sphinxincludegraphics{{zephyr_logo}.png}



\sphinxstylestrong{Note : You may at any time read the book, store it in your ereaders}

The book itself is subject to copyright.

You cannot use the book, or parts of the book into your own publications, without the permission of the author.


\chapter{author:}
\label{\detokenize{copyright:author}}
Jan Jansen
\sphinxhref{mailto:najnesnaj@yahoo.com}{najnesnaj@yahoo.com}


\chapter{LICENSE:}
\label{\detokenize{copyright:license}}
All the software is subject to the Apache 2.0 license (same as zephyr), which is very liberal.


\chapter{Zephyr for the ds\sphinxhyphen{}d6    smartwatch}
\label{\detokenize{content:zephyr-for-the-ds-d6-smartwatch}}\label{\detokenize{content::doc}}
\begin{sphinxVerbatim}[commandchars=\\\{\}]
\PYG{n}{this} \PYG{n}{document} \PYG{n}{describes} \PYG{n}{the} \PYG{n}{installation} \PYG{n}{of} \PYG{n}{zephyr} \PYG{n}{RTOS} \PYG{n}{on} \PYG{n}{the} \PYG{n}{ds}\PYG{o}{\PYGZhy{}}\PYG{n}{d6} \PYG{n}{smartwatch}\PYG{o}{.}


\PYG{n}{It} \PYG{n}{should} \PYG{n}{be} \PYG{n}{applicable} \PYG{n}{on} \PYG{n}{other} \PYG{n}{nordic} \PYG{n}{nrf52832} \PYG{n}{based} \PYG{n}{watches} \PYG{p}{(}\PYG{n}{id107hr} \PYG{n}{plus} \PYG{o}{.}\PYG{o}{.}\PYG{o}{.}\PYG{o}{.}\PYG{p}{)}\PYG{o}{.}
\end{sphinxVerbatim}

\begin{sphinxVerbatim}[commandchars=\\\{\}]
\PYG{n}{the} \PYG{n}{approach} \PYG{o+ow}{in} \PYG{n}{this} \PYG{n}{manual} \PYG{o+ow}{is} \PYG{n}{to} \PYG{n}{get} \PYG{n}{quick} \PYG{n}{results} \PYG{p}{:}
    \PYG{o}{\PYGZhy{}} \PYG{n}{minimal} \PYG{n}{effort} \PYG{n}{install}
    \PYG{o}{\PYGZhy{}} \PYG{k}{try} \PYG{n}{out} \PYG{n}{the} \PYG{n}{samples}
    \PYG{o}{\PYGZhy{}} \PYG{n}{inspire} \PYG{n}{you} \PYG{n}{to} \PYG{n}{modify} \PYG{o+ow}{and} \PYG{n}{enhance}
\end{sphinxVerbatim}
\begin{description}
\item[{suggestion :}] \leavevmode\begin{itemize}
\item {} 
follow the installation instructions

\item {} 
try some examples

\item {} 
try out bluetooth

\item {} 
try out the display

\end{itemize}

\end{description}

\noindent\sphinxincludegraphics{{dsd6}.jpg}


\chapter{Install zephyr}
\label{\detokenize{installation:install-zephyr}}\label{\detokenize{installation::doc}}
\sphinxurl{https://docs.zephyrproject.org/latest/getting\_started/index.html}

the documentation describes an installation process under Ubuntu/macOS/Windows

In order to use the Nordic specific samples, you will need to install their environment, which can co\sphinxhyphen{}exist peacefully.
The changes (updates) lag a month behind the community branch.
\sphinxurl{https://github.com/nrfconnect/sdk-nrf}

you can switch between these environments, by executing in their own directory

\begin{sphinxVerbatim}[commandchars=\\\{\}]
\PYG{g+gp}{\PYGZdl{}} west update
\end{sphinxVerbatim}


\chapter{On the subject of UART}
\label{\detokenize{uart:on-the-subject-of-uart}}\label{\detokenize{uart::doc}}
As I am predisposed of finding out things the hard way, it took me a while to notice why the serial port was only partly functioning :
\begin{itemize}
\item {} 
I used a wrong pin on my dev\sphinxhyphen{}board (id107 HR watch)

\item {} 
I used a USB \sphinxhyphen{} el cheapo \sphinxhyphen{} serial port

\end{itemize}

The nrf52 is an arm chip with TTL levels of 3.3V, so you need a plug that can handle this voltage instead of the usual 5V.

I did not want to spend money and wait for the postman.
I had an stm32f103c8 lying around.

This could be converted in a blackmagicprobe, which has a debugprobe and(!) a serial port. The stm32 is 3.3V compatible. Problem solved, which is quite cool, since it uses up only one USB\sphinxhyphen{}port.


\section{read and learn}
\label{\detokenize{uart:read-and-learn}}
\begin{sphinxVerbatim}[commandchars=\\\{\}]
\PYG{n}{To} \PYG{n+nb}{compile} \PYG{n}{a} \PYG{n}{program} \PYG{k}{with} \PYG{n}{a} \PYG{n}{non} \PYG{n}{standard} \PYG{p}{(}\PYG{o}{=}\PYG{n}{jlink}\PYG{p}{)} \PYG{n}{debugger}\PYG{p}{:}
\PYG{n}{west} \PYG{n}{flash} \PYG{o}{\PYGZhy{}}\PYG{o}{\PYGZhy{}}\PYG{n}{runner} \PYG{n}{blackmagicprobe}
\end{sphinxVerbatim}


\chapter{The black magic probe}
\label{\detokenize{blackmagicprobe:the-black-magic-probe}}\label{\detokenize{blackmagicprobe::doc}}

\section{probes in zephyr}
\label{\detokenize{blackmagicprobe:probes-in-zephyr}}
You can program the nrf52832 with a debuggerprobe.
The standard\sphinxhyphen{}setup is jlink (segger).

/root/zephyrproject/zephyr/boards/arm/id107plus/board.cmake (adapt the runner here)

in our case : instead of jlink specify : blackmagicprobe

The cool thing about this probe that it has a serial port (3.3V) and a debug (upload) port on the same usb\sphinxhyphen{}port.
\begin{itemize}
\item {} 
/dev/ttyACM1 is serial port (pb6 pb7)

\end{itemize}

minicom \sphinxhyphen{}b 115200 \sphinxhyphen{}D /dev/ttyACM1
\begin{itemize}
\item {} 
/dev/ttyACM0 is used as debugger/uploading

\end{itemize}

west debug \textendash{}runner blackmagicprobe
west flash \textendash{}runner blackmagicprobe


\section{howto setup a blackmagicprobe}
\label{\detokenize{blackmagicprobe:howto-setup-a-blackmagicprobe}}
You can buy this probe and support the developers. (make this world a better place)

I bought a “cheapo” “blue pill” stm32 board for future projects …
soldered a 1.8K resistor between 3.3K and PA12

downloaded from \sphinxurl{https://jeelabs.org/docs/software/bmp/}
\sphinxhyphen{} blackmagic.bin (79 ko)
\sphinxhyphen{} blackmagic\_dfu.bin (7 ko)

in jlink : loadbin blackmagic\_dfu.bin 0x8000000 (specify jlink no options …)
switch boot0 or boot1 or whatever
connect usb
in linux
dfu\sphinxhyphen{}util \sphinxhyphen{}v \sphinxhyphen{}R \sphinxhyphen{}d 0483:df11 \sphinxhyphen{}s 0x08002000 \sphinxhyphen{}D blackmagic.bin
(uploading in jlink was a problem cause memory restrictions)

\noindent\sphinxincludegraphics{{blackmagicd6}.jpeg}

(removed boot0 and boot1 connectors on the stm afterwards)

plugged it in the USB port and it pops up (had to enable it first in virtual box usb : black sphere technologies …..)


\chapter{Starting with some basic applications}
\label{\detokenize{basicapplications:starting-with-some-basic-applications}}\label{\detokenize{basicapplications::doc}}
The best way to get a feel of zephyr, is to start building sample applications.

The gpio ports, i2c communication, memory layout, stuff that is particular for the watch is defined in the board definition file.

The provided samples are standard zephyr application, with some minor modifications.


\section{Blinky    example}
\label{\detokenize{basicapplications:blinky-example}}
\begin{sphinxVerbatim}[commandchars=\\\{\}]
\PYG{n}{The} \PYG{n}{watch} \PYG{n}{does} \PYG{o+ow}{not} \PYG{n}{contain} \PYG{n}{a} \PYG{n}{led} \PYG{k}{as} \PYG{n}{such}\PYG{p}{,} \PYG{n}{but} \PYG{n}{it} \PYG{n}{has} \PYG{n}{a} \PYG{n}{heart} \PYG{n}{rate} \PYG{n}{sensor}\PYG{o}{.}

\PYG{n}{Powering} \PYG{n}{the} \PYG{n}{heart} \PYG{n}{rate} \PYG{n}{sensor}\PYG{p}{,} \PYG{n}{lights} \PYG{n}{up} \PYG{n}{the} \PYG{n}{green} \PYG{n}{led}\PYG{o}{.}
\end{sphinxVerbatim}

have a look at the ds\_d6.dts file, here you see the definition of the led.

\sphinxtitleref{building an image, which can be found under the build directory}

see : \DUrole{xref,std,std-ref}{blinky\sphinxhyphen{}sample}

\begin{sphinxVerbatim}[commandchars=\\\{\}]
\PYG{g+gp}{\PYGZdl{}} \PYG{n+nb}{cd} \PYGZti{}/zephyrproject/zephyr
\PYG{g+gp}{\PYGZdl{}} west build \PYGZhy{}p \PYGZhy{}b ds\PYGZus{}d6 samples/basic/blinky
\end{sphinxVerbatim}

once the compilation is completed,  you can find the firmware under :
\textasciitilde{}/zephyrproject/zephyr/build/zephyr/zephyr.bin

you can upload it with:

west flash
or
west flash \textendash{}runner blackmagicprobe (or jlink, or ….)


\chapter{bluetooth (BLE) example}
\label{\detokenize{bluetooth:bluetooth-ble-example}}\label{\detokenize{bluetooth::doc}}
The ds\sphinxhyphen{}d6 uses a Nordic nrf52832 chip, which has BLE functionality build into it.

To test, you can compile a standard application : Eddy Stone.

The watch will behave as a bluetooth beacon, and you should be able to detect it with your smartphone or with bluez under linux.


\section{Using the created bluetooth sample:}
\label{\detokenize{bluetooth:using-the-created-bluetooth-sample}}
I use linux with a bluetoothadapter 4.0.
You will need to install bluez.

\begin{sphinxVerbatim}[commandchars=\\\{\}]
\PYG{g+gp}{\PYGZsh{}}bluetoothctl
\PYG{g+gp}{[bluetooth]\PYGZsh{}}scan on
\end{sphinxVerbatim}

And your Eddy Stone should be visible.

If you have a smartphone, you can download the nrf utilities app from nordic.


\section{Ble Peripheral}
\label{\detokenize{bluetooth:ble-peripheral}}
this example is a demo of the services under bluetooth

first build the image

\begin{sphinxVerbatim}[commandchars=\\\{\}]
\PYG{g+gp}{\PYGZdl{}}  west build \PYGZhy{}p \PYGZhy{}b ds\PYGZus{}d6 samples/bluetooth/peripheral
\end{sphinxVerbatim}

With linux you can have a look using bluetoothctl:

\begin{sphinxVerbatim}[commandchars=\\\{\}]
\PYG{g+gp}{\PYGZsh{}}bluetoothctl
\PYG{g+gp}{[bluetooth]\PYGZsh{}}scan on


\PYG{g+go}{[NEW] Device 60:7C:9E:92:50:C1 Zephyr Peripheral Sample Long}
\PYG{g+go}{once you see your device}
\PYG{g+gp}{[blueooth]\PYGZsh{}}connect \PYG{l+m}{60}:7C:9E:92:50:C1 \PYG{o}{(}the device mac address as displayed\PYG{o}{)}

\PYG{g+go}{then you can already see the services}
\end{sphinxVerbatim}

same thing with the app from nordic, you could try to connect and display value of e.g. heart rate


\section{using Python to read out bluetoothservices}
\label{\detokenize{bluetooth:using-python-to-read-out-bluetoothservices}}
In this repo you will find a python script : readbat.py
In order to use it you need bluez on linux and the python \sphinxtitleref{bluepy} module.

It can be used in conjunction with the peripheral bluetooth demo.
It just reads out the battery level, and prints it.

\begin{sphinxVerbatim}[commandchars=\\\{\}]
\PYG{k+kn}{import} \PYG{n+nn}{binascii}
\PYG{k+kn}{from} \PYG{n+nn}{bluepy}\PYG{n+nn}{.}\PYG{n+nn}{btle} \PYG{k+kn}{import} \PYG{n}{UUID}\PYG{p}{,} \PYG{n}{Peripheral}

\PYG{n}{temp\PYGZus{}uuid} \PYG{o}{=} \PYG{n}{UUID}\PYG{p}{(}\PYG{l+m+mh}{0x2A19}\PYG{p}{)}

\PYG{n}{p} \PYG{o}{=} \PYG{n}{Peripheral}\PYG{p}{(}\PYG{l+s+s2}{\PYGZdq{}}\PYG{l+s+s2}{60:7C:9E:92:50:C1}\PYG{l+s+s2}{\PYGZdq{}}\PYG{p}{,} \PYG{l+s+s2}{\PYGZdq{}}\PYG{l+s+s2}{random}\PYG{l+s+s2}{\PYGZdq{}}\PYG{p}{)}

\PYG{k}{try}\PYG{p}{:}
   \PYG{n}{ch} \PYG{o}{=} \PYG{n}{p}\PYG{o}{.}\PYG{n}{getCharacteristics}\PYG{p}{(}\PYG{n}{uuid}\PYG{o}{=}\PYG{n}{temp\PYGZus{}uuid}\PYG{p}{)}\PYG{p}{[}\PYG{l+m+mi}{0}\PYG{p}{]}
   \PYG{n+nb}{print} \PYG{n}{binascii}\PYG{o}{.}\PYG{n}{b2a\PYGZus{}hex}\PYG{p}{(}\PYG{n}{ch}\PYG{o}{.}\PYG{n}{read}\PYG{p}{(}\PYG{p}{)}\PYG{p}{)}
\PYG{k}{finally}\PYG{p}{:}
    \PYG{n}{p}\PYG{o}{.}\PYG{n}{disconnect}\PYG{p}{(}\PYG{p}{)}
\end{sphinxVerbatim}


\chapter{display (ssd1306)}
\label{\detokenize{display:display-ssd1306}}\label{\detokenize{display:display-sample}}\label{\detokenize{display::doc}}
in order to get the ds d6 display working
you will need to replace the standard display driver by the one provided in the directory drivers
(file ssd1306.c)

for the id107 hr plus you will need the ssd1306.c\sphinxhyphen{}id107hrplus.


\section{Display    example}
\label{\detokenize{display:display-example}}
The display uses the character frame buffer (cfb).

There are two samples which should work out of the box.

check out the  cfb\_shell \sphinxhyphen{} samples, which is really cool.
You will need a serial connection :eg. :  minicom \sphinxhyphen{}b 115200 \sphinxhyphen{}D /dev/ttyACM1.
(just type help to get an overview of commands)

\begin{sphinxVerbatim}[commandchars=\\\{\}]
\PYG{g+gp}{\PYGZdl{}}  cfb init
\PYG{g+gp}{\PYGZdl{}}  cfb print \PYG{l+m}{0} \PYG{l+m}{0} \PYG{l+s+s2}{\PYGZdq{}hello world\PYGZdq{}}
\end{sphinxVerbatim}

\begin{sphinxVerbatim}[commandchars=\\\{\}]
\PYG{g+gp}{\PYGZdl{}}  west build \PYGZhy{}p \PYGZhy{}b ds\PYGZus{}d6 samples/display/cfb
\PYG{g+go}{or}
\PYG{g+gp}{\PYGZdl{}}  west build \PYGZhy{}p \PYGZhy{}b ds\PYGZus{}d6 samples/display/cfb\PYGZus{}shell
\end{sphinxVerbatim}

Displaying special graphics (not just character) is  possible with the lvgl module.


\chapter{LittlevGL Basic Sample}
\label{\detokenize{lvgl:littlevgl-basic-sample}}\label{\detokenize{lvgl:lvgl-sample}}\label{\detokenize{lvgl::doc}}

\section{Overview}
\label{\detokenize{lvgl:overview}}
This sample application displays “Hello World” in the center of the screen.

LittlevGL is a free and open\sphinxhyphen{}source graphics library providing everything you need to create embedded GUI with easy\sphinxhyphen{}to\sphinxhyphen{}use graphical elements, beautiful visual effects and low memory footprint.


\section{Settings}
\label{\detokenize{lvgl:settings}}\begin{itemize}
\item {} 
the lvgl settings happen in the config file (or what it in prj.cfg)

\item {} 
the ssd1306 is a monochrome display

\end{itemize}

CONFIG\_SPI=y
CONFIG\_LVGL\_DISPLAY\_DEV\_NAME=”SSD1306”
CONFIG\_LVGL\_HOR\_RES\_MAX=128
CONFIG\_LVGL\_VER\_RES\_MAX=32
CONFIG\_LVGL\_VDB\_SIZE=64
CONFIG\_LVGL\_BITS\_PER\_PIXEL=1
CONFIG\_LVGL=y
CONFIG\_LVGL\_COLOR\_DEPTH\_1=y
CONFIG\_LVGL\_COLOR\_TRANSP\_GREEN=y
CONFIG\_LVGL\_USE\_API\_EXTENSION\_V6=y
CONFIG\_LVGL\_USE\_API\_EXTENSION\_V7=y
CONFIG\_LVGL\_USE\_THEME\_MONO=y
CONFIG\_LVGL\_THEME\_DEFAULT\_COLOR\_PRIMARY\_WHITE=y
CONFIG\_LVGL\_THEME\_DEFAULT\_COLOR\_SECONDARY\_BLACK=y

CONFIG\_LVGL\_EXT\_CLICK\_AREA\_OFF=y
CONFIG\_LVGL\_USE\_LABEL=y

CONFIG\_LVGL\_LABEL\_DEF\_SCROLL\_SPEED=25
CONFIG\_LVGL\_LABEL\_WAIT\_CHAR\_COUNT=3
\begin{quote}

CONFIG\_LOG\_PROCESS\_TRIGGER\_THRESHOLD=10
CONFIG\_LOG\_PROCESS\_THREAD=y
CONFIG\_LOG\_PROCESS\_THREAD\_SLEEP\_MS=1000
CONFIG\_LOG\_PROCESS\_THREAD\_STACK\_SIZE=768
CONFIG\_LOG\_BUFFER\_SIZE=1024
CONFIG\_LOG\_DETECT\_MISSED\_STRDUP=y
CONFIG\_LOG\_STRDUP\_MAX\_STRING=400
CONFIG\_LOG\_STRDUP\_BUF\_COUNT=32
\end{quote}


\section{References}
\label{\detokenize{lvgl:references}}
\sphinxurl{https://docs.littlevgl.com/en/html/index.html}

LittlevGL Web Page: \sphinxurl{https://littlevgl.com/}


\chapter{Current Time Service}
\label{\detokenize{current-time:current-time-service}}\label{\detokenize{current-time::doc}}
\sphinxurl{https://www.bluetooth.com/specifications/gatt/services/}
\sphinxurl{https://www.bluetooth.com/specifications/gatt/characteristics/}
0x1805 current time service
0x2A2B current time characteristic


\section{Requirements:}
\label{\detokenize{current-time:requirements}}\begin{description}
\item[{You need :}] \leavevmode\begin{itemize}
\item {} \begin{description}
\item[{a CTS server (use of bluez on linux explained)}] \leavevmode\begin{itemize}
\item {} 
start the CTS service (python script)

\item {} 
connect to the CTS client

\end{itemize}

\end{description}

\item {} 
a CTS client (the pinetime watch)

\end{itemize}

\end{description}


\section{BLE Peripheral CTS sample for zephyr}
\label{\detokenize{current-time:ble-peripheral-cts-sample-for-zephyr}}
This example demonstrates the basic usage of the current time service.
It is based on the \sphinxurl{https://github.com/Dejvino/pinetime-hermes-firmware}.
It starts advertising it’s UUID, and you can connect to it.
Once connected, it will read the time from your CTS server (bluez on linux running the gatt\sphinxhyphen{}cts\sphinxhyphen{}server script in my case)

first build the image

\begin{sphinxVerbatim}[commandchars=\\\{\}]
\PYG{g+gp}{\PYGZdl{}}  west build \PYGZhy{}p \PYGZhy{}b ds\PYGZus{}d6 samples/bluetooth/peripheral\PYGZhy{}cts
\end{sphinxVerbatim}


\section{Using bluez on linux to connect}
\label{\detokenize{current-time:using-bluez-on-linux-to-connect}}\begin{description}
\item[{The ds\_d6 zephyr sample behaves as a peripheral:}] \leavevmode\begin{itemize}
\item {} 
first of all start the cts service

\end{itemize}
\begin{quote}

\sphinxhyphen{}connect to the ds\_d6 with bluetoothctl
\end{quote}

\end{description}

Using bluetoothctl:

\begin{sphinxVerbatim}[commandchars=\\\{\}]
\PYG{g+gp}{\PYGZsh{}}bluetoothctl
\PYG{g+gp}{[bluetooth]\PYGZsh{}}scan on


\PYG{g+go}{[NEW] Device 60:7C:9E:92:50:C1 Zephyr Peripheral Sample Long}
\PYG{g+go}{once you see your device}
\PYG{g+gp}{[blueooth]\PYGZsh{}}connect \PYG{l+m}{60}:7C:9E:92:50:C1 \PYG{o}{(}the device mac address as displayed\PYG{o}{)}
\end{sphinxVerbatim}


\section{Howto use Bluez on linux to set up a time service}
\label{\detokenize{current-time:howto-use-bluez-on-linux-to-set-up-a-time-service}}
Within the bluez source distribution there is an example GATT (Generic Attribute Profile)server. It advertises some standard service such as heart rate, battery …
Koen zandberg adapted this script, so it advertises the current time :
\sphinxurl{https://github.com/bosmoment/gatt-cts/blob/master/gatt-cts-server.py}

You might have to install extra packages:

\begin{sphinxVerbatim}[commandchars=\\\{\}]
\PYG{g+go}{apt\PYGZhy{}get install python\PYGZhy{}dbus}
\PYG{g+go}{apt\PYGZhy{}get install python\PYGZhy{}gi}
\PYG{g+go}{apt\PYGZhy{}get install python\PYGZhy{}gobject}
\end{sphinxVerbatim}


\section{Howto use Android to set up a time service}
\label{\detokenize{current-time:howto-use-android-to-set-up-a-time-service}}
As soon as a device is bonded, Pinetime will look for a CTS server (Current Time Service) on the connected device.
Here is how to do it with an Android smartphone running NRFConnect:

Build and program the firmware on the Pinetime Install NRFConnect (\sphinxurl{https://www.nordicsemi.com/Software-and-Tools/Development-Tools/nRF-Connect-for-desktop})

Start NRFConnect and create a CTS server : Tap the hamburger button on the top left and select “Configure GATT server” Tap “Add service” on the bottom Select server configuration “Current Time Service” and tap OK Go back to the main screen and scan for BLE devices. A device called “PineTime” should appear Tap the button “Connect” next to the PineTime device. It should connect to the PineTime and switch to a new tab. On this tab, on the top right, there is a 3 dots button. Tap on it and select Bond. The bonding process begins, and if it is sucessful, the PineTime should update its time and display it on the screen.


\chapter{Firmware Over The Air (FOTA)}
\label{\detokenize{fota/fota:firmware-over-the-air-fota}}\label{\detokenize{fota/fota:fota}}\label{\detokenize{fota/fota::doc}}

\section{Serial Device Firmware Upgrade}
\label{\detokenize{fota/mcuboot:serial-device-firmware-upgrade}}\label{\detokenize{fota/mcuboot:mcuboot}}\label{\detokenize{fota/mcuboot::doc}}
My main aim is to get this working.
It has a small footprint, since no need for bluetooth.

I have not(!) been successfull.

west build \sphinxhyphen{}b ds\_d6 samples/display/cfb \textendash{} \sphinxhyphen{}DOVERLAY\_CONFIG=’overlay\sphinxhyphen{}serial.conf;overlay\sphinxhyphen{}fs.conf’  \sphinxhyphen{}DCONFIG\_MCUBOOT\_SIGNATURE\_KEY\_FILE=”../bootloader/mcuboot/root\sphinxhyphen{}rsa\sphinxhyphen{}2048.pem”


\section{Wireless Device Firmware Upgrade}
\label{\detokenize{fota/mcuboot:wireless-device-firmware-upgrade}}

\subsection{Overview}
\label{\detokenize{fota/mcuboot:overview}}
In order to perform a FOTA (firmware over the air) update on zephyr you need 2 basic components:
\begin{itemize}
\item {} 
MCUboot   (a bootloader)

\item {} 
SMP Server (a bluetooth service)

\end{itemize}


\section{MCUboot with zephyr}
\label{\detokenize{fota/mcuboot:mcuboot-with-zephyr}}\label{\detokenize{fota/mcuboot:id1}}
west build \sphinxhyphen{}b ds\_d6 \sphinxhyphen{}s ../bootloader/mcuboot/boot/zephyr \sphinxhyphen{}d build\sphinxhyphen{}mcuboot
west flash \sphinxhyphen{}d build\sphinxhyphen{}mcuboot \textendash{}runner blackmagicprobe

Some additional configuration is required to build applications for MCUboot.

This is handled internally by the Zephyr configuration system and is wrapped
in the \sphinxtitleref{CONFIG\_BOOTLOADER\_MCUBOOT} Kconfig variable, which must be enabled in
the application’s \sphinxtitleref{prj.conf} file.

west build \sphinxhyphen{}b ds\_d6 samples/subsys/console/echo \sphinxhyphen{}d build\sphinxhyphen{}hello\sphinxhyphen{}signed \sphinxhyphen{}DCONFIG\_MCUBOOT\_SIGNATURE\_KEY\_FILE=”../bootloader/mcuboot/root\sphinxhyphen{}rsa\sphinxhyphen{}2048.pem”
west flash \sphinxhyphen{}d build\sphinxhyphen{}hello\sphinxhyphen{}signed \textendash{}runner blackmagicprobe

another example with overlays:
west build \sphinxhyphen{}b ds\_d6 samples/subsys/mgmt/mcumgr/smp\_svr \textendash{} \sphinxhyphen{}DOVERLAY\_CONFIG=’overlay\sphinxhyphen{}serial.conf;overlay\sphinxhyphen{}fs.conf’

mcumgr \sphinxhyphen{}ldebug \sphinxhyphen{}t 60 \textendash{}conntype=serial \textendash{}connstring=’dev=/dev/ttyACM1,baud=115200’ image list

In order to upgrade to an image (or even boot it, if
\sphinxtitleref{MCUBOOT\_VALIDATE\_PRIMARY\_SLOT} is enabled), the images must be signed.

To make development easier, MCUboot is distributed with some example
keys.  It is important to stress that these should never be used for
production, since the private key is publicly available in this
repository.  See below on how to make your own signatures.

Images can be signed with the \sphinxtitleref{scripts/imgtool.py} script.  It is best
to look at \sphinxtitleref{samples/zephyr/Makefile} for examples on how to use this.

west flash knows where to put the application (you do not have to specify the address of the slot)


\section{Partitions}
\label{\detokenize{fota/partitions:partitions}}\label{\detokenize{fota/partitions:signing}}\label{\detokenize{fota/partitions::doc}}
\begin{sphinxVerbatim}[commandchars=\\\{\}]
\PYG{g+go}{have a look at boards/ds\PYGZus{}d6.dts}
\end{sphinxVerbatim}


\subsection{Defining partitions for MCUboot}
\label{\detokenize{fota/partitions:defining-partitions-for-mcuboot}}
The first step required for Zephyr is making sure your board has flash
partitions defined in its device tree. These partitions are:
\begin{itemize}
\item {} 
\sphinxtitleref{boot\_partition}: for MCUboot itself

\item {} 
\sphinxtitleref{image\_0\_primary\_partition}: the primary slot of Image 0

\item {} 
\sphinxtitleref{image\_0\_secondary\_partition}: the secondary slot of Image 0

\item {} 
\sphinxtitleref{scratch\_partition}: the scratch slot

\end{itemize}


\subsection{Remark}
\label{\detokenize{fota/partitions:remark}}
the ds\_d6 has no nor flash. Having 2 slots limits the size of you program!
You can also use 1 slot. (is less secure of course)


\section{Signing an application}
\label{\detokenize{fota/signing:signing-an-application}}\label{\detokenize{fota/signing:signing}}\label{\detokenize{fota/signing::doc}}
Never use the default public key, but generate your own!

In order to improve the security, only signed images can be uploaded.

There is a public and private key.
The Bootloader is compiled with the public key.
Each time you want to upload firmware, you have to sign it with a private key.

\sphinxstylestrong{NOTE: it is important to keep the private key hidden}


\subsection{Generating a new keypair}
\label{\detokenize{fota/signing:generating-a-new-keypair}}
Generating a keypair with imgtool is a matter of running the keygen
subcommand:

\begin{sphinxVerbatim}[commandchars=\\\{\}]
\PYG{g+gp}{\PYGZdl{}} ./scripts/imgtool.py keygen \PYGZhy{}k mykey.pem \PYGZhy{}t rsa\PYGZhy{}2048
\end{sphinxVerbatim}


\subsection{Extracting the public key}
\label{\detokenize{fota/signing:extracting-the-public-key}}
The generated keypair above contains both the public and the private
key.  It is necessary to extract the public key and insert it into the
bootloader.

\begin{sphinxVerbatim}[commandchars=\\\{\}]
\PYG{g+gp}{\PYGZdl{}} ./scripts/imgtool.py getpub \PYGZhy{}k mykey.pem
\end{sphinxVerbatim}

This will output the public key as a C array that can be dropped
directly into the \sphinxtitleref{keys.c} file.


\subsection{Example}
\label{\detokenize{fota/signing:example}}
sign the compiled zephyr.bin firmware with the root\sphinxhyphen{}rsa\sphinxhyphen{}2048.pem, private key:

\begin{sphinxVerbatim}[commandchars=\\\{\}]
\PYG{g+go}{imgtool.py sign \PYGZhy{}\PYGZhy{}key ../../root\PYGZhy{}rsa\PYGZhy{}2048.pem \PYGZbs{}}
\PYG{g+go}{    \PYGZhy{}\PYGZhy{}header\PYGZhy{}size 0x200 \PYGZbs{}}
\PYG{g+go}{    \PYGZhy{}\PYGZhy{}align 8 \PYGZbs{}}
\PYG{g+go}{    \PYGZhy{}\PYGZhy{}version 1.2 \PYGZbs{}}
\PYG{g+go}{    \PYGZhy{}\PYGZhy{}slot\PYGZhy{}size 0x60000 \PYGZbs{}}
\PYG{g+go}{    ../mcuboot/samples/zephyr/build/ds\PYGZus{}d6/hello1/zephyr/zephyr.bin \PYGZbs{}}
\PYG{g+go}{    signed\PYGZhy{}hello1.bin}
\end{sphinxVerbatim}


\section{SMP Server Sample}
\label{\detokenize{fota/smp_svr:smp-server-sample}}\label{\detokenize{fota/smp_svr:smp-svr-sample}}\label{\detokenize{fota/smp_svr::doc}}

\subsection{Overview}
\label{\detokenize{fota/smp_svr:overview}}
This sample application implements a Simple Management Protocol (SMP) server.
SMP is a basic transfer encoding for use with the MCUmgr management protocol.

This sample application supports the following mcumgr transports by default:
\begin{itemize}
\item {} 
Shell

\item {} 
Bluetooth

\end{itemize}


\subsection{Requirements}
\label{\detokenize{fota/smp_svr:requirements}}
In order to communicate with the smp server sample installed on your pinetime, you need mcumgr.

Here is a procedure to install mcumgr on a raspberry pi  (or similar)

It is written in the go\sphinxhyphen{}language. You need to adapt the path :   PATH=\$PATH:/root/go/bin.


\subsection{Building and Running}
\label{\detokenize{fota/smp_svr:building-and-running}}
The sample will let you manage the pinetime over bluetooth. (via SMP protocol)

There are slot0 and slot1 which can both contain firmware.

Suppose you switch from slot0 to slot1, you still want to be able to communicate.

So both slots need smp\_svr software!


\subsubsection{Step 1: Build smp\_svr}
\label{\detokenize{fota/smp_svr:step-1-build-smp-svr}}
\sphinxcode{\sphinxupquote{smp\_svr}} can be built for the nRF52 as follows:

\sphinxstylestrong{NOTE: to perform a firmware update over the air, you have to build a second sample}


\subsubsection{Step 2: Sign the image}
\label{\detokenize{fota/smp_svr:step-2-sign-the-image}}
Using MCUboot’s \sphinxcode{\sphinxupquote{imgtool.py}} script, sign the \sphinxcode{\sphinxupquote{zephyr.(bin|hex)}}
file you built in Step 3. In the below example, the MCUboot repo is located at
\sphinxcode{\sphinxupquote{\textasciitilde{}/src/mcuboot}}.

\begin{sphinxVerbatim}[commandchars=\\\{\}]
\PYG{g+go}{\PYGZti{}/src/mcuboot/scripts/imgtool.py sign \PYGZbs{}}
\PYG{g+go}{     \PYGZhy{}\PYGZhy{}key \PYGZti{}/src/mcuboot/root\PYGZhy{}rsa\PYGZhy{}2048.pem \PYGZbs{}}
\PYG{g+go}{     \PYGZhy{}\PYGZhy{}header\PYGZhy{}size 0x200 \PYGZbs{}}
\PYG{g+go}{     \PYGZhy{}\PYGZhy{}align 8 \PYGZbs{}}
\PYG{g+go}{     \PYGZhy{}\PYGZhy{}version 1.0 \PYGZbs{}}
\PYG{g+go}{     \PYGZhy{}\PYGZhy{}slot\PYGZhy{}size \PYGZlt{}image\PYGZhy{}slot\PYGZhy{}size\PYGZgt{} \PYGZbs{}}
\PYG{g+go}{     \PYGZlt{}path\PYGZhy{}to\PYGZhy{}zephyr.(bin|hex)\PYGZgt{} signed.(bin|hex)}
\end{sphinxVerbatim}

The above command creates an image file called \sphinxcode{\sphinxupquote{signed.(bin|hex)}} in the
current directory.


\subsubsection{Step 3: Flash the smp\_svr image}
\label{\detokenize{fota/smp_svr:step-3-flash-the-smp-svr-image}}
Upload the bin\sphinxhyphen{}file from Step 2 to image slot\sphinxhyphen{}0.
For the pinetime, slot\sphinxhyphen{}0 is located at address \sphinxcode{\sphinxupquote{0xc000}}.

\begin{sphinxVerbatim}[commandchars=\\\{\}]
\PYG{g+go}{in openocd : program zephyr.bin 0xc000}
\end{sphinxVerbatim}


\subsubsection{Step 4: Run it!}
\label{\detokenize{fota/smp_svr:step-4-run-it}}
\begin{sphinxadmonition}{note}{Note:}
If you haven’t installed \sphinxcode{\sphinxupquote{mcumgr}} yet, then do so by following the
instructions in the \DUrole{xref,std,std-ref}{mcumgr\_cli} section of the Management subsystem
documentation.
\end{sphinxadmonition}

The \sphinxcode{\sphinxupquote{smp\_svr}} app is ready to run.  Just reset your board and test the app
with the \sphinxcode{\sphinxupquote{mcumgr}} command\sphinxhyphen{}line tool’s \sphinxcode{\sphinxupquote{echo}} functionality, which will
send a string to the remote target device and have it echo it back:

\begin{sphinxVerbatim}[commandchars=\\\{\}]
\PYG{g+go}{sudo mcumgr \PYGZhy{}\PYGZhy{}conntype ble \PYGZhy{}\PYGZhy{}connstring ctlr\PYGZus{}name=hci0,peer\PYGZus{}name=\PYGZsq{}Zephyr\PYGZsq{} echo hello}
\PYG{g+go}{hello}
\end{sphinxVerbatim}


\subsubsection{Step 5: Device Firmware Upgrade}
\label{\detokenize{fota/smp_svr:step-5-device-firmware-upgrade}}
Now that the SMP server is running on your pinetime, you are able to communicate
with it using \sphinxtitleref{mcumgr}.

You might want to test “OTA DFU”, or Over\sphinxhyphen{}The\sphinxhyphen{}Air Device Firmware Upgrade.

To do this, build a second sample (following the steps below) to verify
it is sent over the air and properly flashed into slot\sphinxhyphen{}1, and then
swapped into slot\sphinxhyphen{}0 by MCUboot.

\begin{sphinxVerbatim}[commandchars=\\\{\}]
\PYG{o}{*} \PYG{n}{Build} \PYG{n}{a} \PYG{n}{second} \PYG{n}{sample}
\PYG{o}{*} \PYG{n}{Sign} \PYG{n}{the} \PYG{n}{second} \PYG{n}{sample}
\PYG{o}{*} \PYG{n}{Upload} \PYG{n}{the} \PYG{n}{image} \PYG{n}{over} \PYG{n}{BLE}
\end{sphinxVerbatim}

Now we are ready to send or upload the image over BLE to the target remote
device.

\begin{sphinxVerbatim}[commandchars=\\\{\}]
\PYG{g+go}{sudo mcumgr \PYGZhy{}\PYGZhy{}conntype ble \PYGZhy{}\PYGZhy{}connstring ctlr\PYGZus{}name=hci0,peer\PYGZus{}name=\PYGZsq{}Zephyr\PYGZsq{} image upload signed.bin}
\end{sphinxVerbatim}

If all goes well the image will now be stored in slot\sphinxhyphen{}1, ready to be swapped
into slot\sphinxhyphen{}0 and executed.

\begin{sphinxadmonition}{note}{Note:}
At the beginning of the upload process, the target might start erasing
the image slot, taking several dozen seconds for some targets.  This might
cause an NMP timeout in the management protocol tool. Use the
\sphinxcode{\sphinxupquote{\sphinxhyphen{}t \textless{}timeout\sphinxhyphen{}in\sphinxhyphen{}seconds}} option to increase the response timeout for the
\sphinxcode{\sphinxupquote{mcumgr}} command line tool if this occurs.
\end{sphinxadmonition}


\paragraph{List the images}
\label{\detokenize{fota/smp_svr:list-the-images}}
We can now obtain a list of images (slot\sphinxhyphen{}0 and slot\sphinxhyphen{}1) present in the remote
target device by issuing the following command:

\begin{sphinxVerbatim}[commandchars=\\\{\}]
\PYG{g+go}{sudo mcumgr \PYGZhy{}\PYGZhy{}conntype ble \PYGZhy{}\PYGZhy{}connstring ctlr\PYGZus{}name=hci0,peer\PYGZus{}name=\PYGZsq{}Zephyr\PYGZsq{} image list}
\end{sphinxVerbatim}

This should print the status and hash values of each of the images present.


\paragraph{Test the image}
\label{\detokenize{fota/smp_svr:test-the-image}}
In order to instruct MCUboot to swap the images we need to test the image first,
making sure it boots:

\begin{sphinxVerbatim}[commandchars=\\\{\}]
\PYG{g+go}{sudo mcumgr \PYGZhy{}\PYGZhy{}conntype ble \PYGZhy{}\PYGZhy{}connstring ctlr\PYGZus{}name=hci0,peer\PYGZus{}name=\PYGZsq{}Zephyr\PYGZsq{} image test \PYGZlt{}hash of slot\PYGZhy{}1 image\PYGZgt{}}
\end{sphinxVerbatim}

Now MCUBoot will swap the image on the next reset.


\paragraph{Reset remotely}
\label{\detokenize{fota/smp_svr:reset-remotely}}
We can reset the device remotely to observe (use the console output) how
MCUboot swaps the images:

\begin{sphinxVerbatim}[commandchars=\\\{\}]
\PYG{g+go}{sudo mcumgr \PYGZhy{}\PYGZhy{}conntype ble \PYGZhy{}\PYGZhy{}connstring ctlr\PYGZus{}name=hci0,peer\PYGZus{}name=\PYGZsq{}Zephyr\PYGZsq{} reset}
\end{sphinxVerbatim}

Upon reset MCUboot will swap slot\sphinxhyphen{}0 and slot\sphinxhyphen{}1.

You can confirm the new image and make the swap permanent by using this command:

\begin{sphinxVerbatim}[commandchars=\\\{\}]
\PYG{g+go}{sudo mcumgr \PYGZhy{}\PYGZhy{}conntype ble \PYGZhy{}\PYGZhy{}connstring ctlr\PYGZus{}name=hci0,peer\PYGZus{}name=\PYGZsq{}Zephyr\PYGZsq{} image confirm}
\end{sphinxVerbatim}

\sphinxstylestrong{Note} that if you try to send the very same image that is already flashed in
slot\sphinxhyphen{}0 then the procedure will not complete successfully since the hash values
for both slots will be identical.


\chapter{Samples and Demos}
\label{\detokenize{samples/index:samples-and-demos}}\label{\detokenize{samples/index:id1}}\label{\detokenize{samples/index::doc}}
In each sample directory is a Readme file.
This is just a collection of them.


\section{Character frame buffer}
\label{\detokenize{samples/cfb/README:character-frame-buffer}}\label{\detokenize{samples/cfb/README:character-frame-buffer-sample}}\label{\detokenize{samples/cfb/README::doc}}

\subsection{Overview}
\label{\detokenize{samples/cfb/README:overview}}
This sample displays character strings using the Character Frame Buffer
(CFB) subsystem framework.


\subsection{Building and Running}
\label{\detokenize{samples/cfb/README:building-and-running}}
build the application: west build \sphinxhyphen{}p \sphinxhyphen{}b ds\_d6 samples/display/cfb

on unix : \#minicom \sphinxhyphen{}b 115200 \sphinxhyphen{}D /dev/ttyACM1

you get a shell and you can type help

to display something on the screen :
cfb init
cfb invert
cfp print 0 0 “hello world”


\section{Bluetooth: Peripheral UART}
\label{\detokenize{samples/peripheral_uart/README:bluetooth-peripheral-uart}}\label{\detokenize{samples/peripheral_uart/README:peripheral-uart}}\label{\detokenize{samples/peripheral_uart/README::doc}}
for this you will need the nordic branch of zephyr : nrfconnect/sdk\sphinxhyphen{}zephyr


\section{TODO}
\label{\detokenize{samples/peripheral_uart/README:todo}}
this samples prints data on serial port on the bluetooth service (and the other way around)

it would be cool if you could use cfb (character frame buffer) as well. (send messages to screen)

\begin{sphinxShadowBox}
\begin{itemize}
\item {} 
\phantomsection\label{\detokenize{samples/peripheral_uart/README:id3}}{\hyperref[\detokenize{samples/peripheral_uart/README:overview}]{\sphinxcrossref{Overview}}}

\item {} 
\phantomsection\label{\detokenize{samples/peripheral_uart/README:id4}}{\hyperref[\detokenize{samples/peripheral_uart/README:debugging}]{\sphinxcrossref{Debugging}}}

\item {} 
\phantomsection\label{\detokenize{samples/peripheral_uart/README:id5}}{\hyperref[\detokenize{samples/peripheral_uart/README:building-and-running}]{\sphinxcrossref{Building and running}}}
\begin{itemize}
\item {} 
\phantomsection\label{\detokenize{samples/peripheral_uart/README:id6}}{\hyperref[\detokenize{samples/peripheral_uart/README:testing}]{\sphinxcrossref{Testing}}}

\end{itemize}

\item {} 
\phantomsection\label{\detokenize{samples/peripheral_uart/README:id7}}{\hyperref[\detokenize{samples/peripheral_uart/README:dependencies}]{\sphinxcrossref{Dependencies}}}

\end{itemize}
\end{sphinxShadowBox}

The Peripheral UART sample demonstrates how to use the \DUrole{xref,std,std-ref}{nus\_service\_readme}.
It uses the NUS service to send data back and forth between a UART connection and a Bluetooth LE connection, emulating a serial port over Bluetooth LE.


\subsection{Overview}
\label{\detokenize{samples/peripheral_uart/README:overview}}
When connected, the sample forwards any data received on the RX pin of the UART 1 peripheral to the Bluetooth LE unit.
On Nordic Semiconductor’s development kits, the UART 1 peripheral is typically gated through the SEGGER chip to a USB CDC virtual serial port.

Any data sent from the Bluetooth LE unit is sent out of the UART 1 peripheral’s TX pin.


\subsection{Debugging}
\label{\detokenize{samples/peripheral_uart/README:debugging}}\label{\detokenize{samples/peripheral_uart/README:peripheral-uart-debug}}
In this sample, the UART console is used to send and read data over the NUS service.
Debug messages are not displayed in this UART console.
Instead, they are printed by the RTT logger.

If you want to view the debug messages, follow the procedure in \DUrole{xref,std,std-ref}{testing\_rtt\_connect}.


\subsection{Building and running}
\label{\detokenize{samples/peripheral_uart/README:building-and-running}}

\subsubsection{Testing}
\label{\detokenize{samples/peripheral_uart/README:testing}}\label{\detokenize{samples/peripheral_uart/README:peripheral-uart-testing}}

\subsection{Dependencies}
\label{\detokenize{samples/peripheral_uart/README:dependencies}}
This sample uses the following {\color{red}\bfseries{}|NCS|} libraries:
\begin{itemize}
\item {} 
\DUrole{xref,std,std-ref}{nus\_service\_readme}

\end{itemize}

In addition, it uses the following Zephyr libraries:
\begin{itemize}
\item {} 
\sphinxcode{\sphinxupquote{include/zephyr/types.h}}

\item {} 
\sphinxcode{\sphinxupquote{boards/arm/nrf*/board.h}}

\item {} 
\DUrole{xref,std,std-ref}{zephyr:kernel\_api}:
\begin{itemize}
\item {} 
\sphinxcode{\sphinxupquote{include/kernel.h}}

\end{itemize}

\item {} 
\DUrole{xref,std,std-ref}{zephyr:api\_peripherals}:
\begin{itemize}
\item {} 
\sphinxcode{\sphinxupquote{incude/gpio.h}}

\item {} 
\sphinxcode{\sphinxupquote{include/uart.h}}

\end{itemize}

\item {} 
\DUrole{xref,std,std-ref}{zephyr:bluetooth\_api}:
\begin{itemize}
\item {} 
\sphinxcode{\sphinxupquote{include/bluetooth/bluetooth.h}}

\item {} 
\sphinxcode{\sphinxupquote{include/bluetooth/gatt.h}}

\item {} 
\sphinxcode{\sphinxupquote{include/bluetooth/hci.h}}

\item {} 
\sphinxcode{\sphinxupquote{include/bluetooth/uuid.h}}

\end{itemize}

\end{itemize}


\section{Bluetooth: NUS shell transport}
\label{\detokenize{samples/shell_bt_nus/README:bluetooth-nus-shell-transport}}\label{\detokenize{samples/shell_bt_nus/README:shell-bt-nus}}\label{\detokenize{samples/shell_bt_nus/README::doc}}
for this, you will need the nordic branch of zephyr (\sphinxurl{https://github.com/nrfconnect/sdk-zephyr})

The Nordic UART Service (NUS) shell transport sample demonstrates how to use the \DUrole{xref,std,std-ref}{shell\_bt\_nus\_readme} to receive shell commands from a remote device.


\section{TODO}
\label{\detokenize{samples/shell_bt_nus/README:todo}}
You can use the shell to type messages on the OLED screen (cfb character frame buffer)
You can use bluetooth shell to type messages on OLED screen.

It would be really cool to use bt\sphinxhyphen{}shell to type message on uart (and the other way around)

I experimented with bt\sphinxhyphen{}shell and here you can use gatt to type characters on bluetooth.


\subsection{Overview}
\label{\detokenize{samples/shell_bt_nus/README:overview}}
When the connection is established, you can connect to the sample through the \DUrole{xref,std,std-ref}{nus\_service\_readme} by using a host application.
You can then send shell commands, that are executed on the device that runs the sample, and see the logs.
See \DUrole{xref,std,std-ref}{shell\_bt\_nus\_host\_tools} for more information about the host tools available, in {\color{red}\bfseries{}|NCS|}, for communicating with the sample.


\subsection{Building and running}
\label{\detokenize{samples/shell_bt_nus/README:building-and-running}}

\subsubsection{Testing using the BLE Console}
\label{\detokenize{samples/shell_bt_nus/README:testing-using-the-ble-console}}
See \DUrole{xref,std,std-ref}{ble\_console\_readme} for more information on how to test the sample using the BLE Console.


\subsection{Dependencies}
\label{\detokenize{samples/shell_bt_nus/README:dependencies}}
This sample uses the following {\color{red}\bfseries{}|NCS|} libraries:
\begin{itemize}
\item {} 
\DUrole{xref,std,std-ref}{shell\_bt\_nus\_readme}

\item {} 
\DUrole{xref,std,std-ref}{nus\_service\_readme}

\end{itemize}

In addition, it uses the following Zephyr libraries:
\begin{itemize}
\item {} 
\DUrole{xref,std,std-ref}{zephyr:bluetooth\_api}:
\begin{itemize}
\item {} 
\sphinxcode{\sphinxupquote{include/bluetooth/bluetooth.h}}

\item {} 
\sphinxcode{\sphinxupquote{include/bluetooth/hci.h}}

\item {} 
\sphinxcode{\sphinxupquote{include/bluetooth/uuid.h}}

\item {} 
\sphinxcode{\sphinxupquote{include/bluetooth/gatt.h}}

\item {} 
\sphinxcode{\sphinxupquote{samples/bluetooth/gatt/bas.h}}

\end{itemize}

\item {} 
\DUrole{xref,std,std-ref}{zephyr:logging\_api}

\end{itemize}


\section{LittlevGL Basic Sample}
\label{\detokenize{samples/lvgl2/README:littlevgl-basic-sample}}\label{\detokenize{samples/lvgl2/README:lvgl-sample}}\label{\detokenize{samples/lvgl2/README::doc}}

\subsection{Overview}
\label{\detokenize{samples/lvgl2/README:overview}}
This sample application displays “Hello World” in the center of the screen.


\subsection{Requirements}
\label{\detokenize{samples/lvgl2/README:requirements}}

\subsection{Building and Running}
\label{\detokenize{samples/lvgl2/README:building-and-running}}
copy sample to your zephyr directory ../samples/display

west build \sphinxhyphen{}p \sphinxhyphen{}b ds\_d6 \sphinxhyphen{}d build\sphinxhyphen{}lvgl samples/display/lvgl2


\subsection{References}
\label{\detokenize{samples/lvgl2/README:references}}

\chapter{Menuconfig}
\label{\detokenize{menuconfig:menuconfig}}\label{\detokenize{menuconfig::doc}}

\section{Zephyr is like linux}
\label{\detokenize{menuconfig:zephyr-is-like-linux}}
Once you have build a sample, you can add extra stuff by using the configuration menu.

\begin{sphinxVerbatim}[commandchars=\\\{\}]
\PYG{g+gp}{\PYGZdl{}} west build \PYGZhy{}t menuconfig
\end{sphinxVerbatim}

\begin{sphinxVerbatim}[commandchars=\\\{\}]
    \PYG{n}{Modules}  \PYG{o}{\PYGZhy{}}\PYG{o}{\PYGZhy{}}\PYG{o}{\PYGZhy{}}\PYG{o}{\PYGZgt{}}
    \PYG{n}{Board} \PYG{n}{Selection} \PYG{p}{(}\PYG{n}{nRF52832}\PYG{o}{\PYGZhy{}}\PYG{n}{MDK}\PYG{p}{)}  \PYG{o}{\PYGZhy{}}\PYG{o}{\PYGZhy{}}\PYG{o}{\PYGZhy{}}\PYG{o}{\PYGZgt{}}
    \PYG{n}{Board} \PYG{n}{Options}  \PYG{o}{\PYGZhy{}}\PYG{o}{\PYGZhy{}}\PYG{o}{\PYGZhy{}}\PYG{o}{\PYGZgt{}}
    \PYG{n}{SoC}\PYG{o}{/}\PYG{n}{CPU}\PYG{o}{/}\PYG{n}{Configuration} \PYG{n}{Selection} \PYG{p}{(}\PYG{n}{Nordic} \PYG{n}{Semiconductor} \PYG{n}{nRF52} \PYG{n}{series} \PYG{n}{MCU}\PYG{p}{)}  \PYG{o}{\PYGZhy{}}\PYG{o}{\PYGZhy{}}\PYG{o}{\PYGZhy{}}\PYG{o}{\PYGZgt{}}
    \PYG{n}{Hardware} \PYG{n}{Configuration}  \PYG{o}{\PYGZhy{}}\PYG{o}{\PYGZhy{}}\PYG{o}{\PYGZhy{}}\PYG{o}{\PYGZgt{}}
    \PYG{n}{ARM} \PYG{n}{Options}  \PYG{o}{\PYGZhy{}}\PYG{o}{\PYGZhy{}}\PYG{o}{\PYGZhy{}}\PYG{o}{\PYGZgt{}}
    \PYG{n}{Architecture} \PYG{p}{(}\PYG{n}{ARM} \PYG{n}{architecture}\PYG{p}{)}  \PYG{o}{\PYGZhy{}}\PYG{o}{\PYGZhy{}}\PYG{o}{\PYGZhy{}}\PYG{o}{\PYGZgt{}}
    \PYG{n}{General} \PYG{n}{Architecture} \PYG{n}{Options}  \PYG{o}{\PYGZhy{}}\PYG{o}{\PYGZhy{}}\PYG{o}{\PYGZhy{}}\PYG{o}{\PYGZgt{}}
\PYG{p}{[} \PYG{p}{]} \PYG{n}{Floating} \PYG{n}{point}  \PYG{o}{\PYGZhy{}}\PYG{o}{\PYGZhy{}}\PYG{o}{\PYGZhy{}}\PYG{o}{\PYGZhy{}}
    \PYG{n}{General} \PYG{n}{Kernel} \PYG{n}{Options}  \PYG{o}{\PYGZhy{}}\PYG{o}{\PYGZhy{}}\PYG{o}{\PYGZhy{}}\PYG{o}{\PYGZgt{}}
    \PYG{n}{C} \PYG{n}{Library}  \PYG{o}{\PYGZhy{}}\PYG{o}{\PYGZhy{}}\PYG{o}{\PYGZhy{}}\PYG{o}{\PYGZgt{}}
    \PYG{n}{Additional} \PYG{n}{libraries}  \PYG{o}{\PYGZhy{}}\PYG{o}{\PYGZhy{}}\PYG{o}{\PYGZhy{}}\PYG{o}{\PYGZgt{}}
\PYG{p}{[}\PYG{o}{*}\PYG{p}{]} \PYG{n}{Bluetooth}  \PYG{o}{\PYGZhy{}}\PYG{o}{\PYGZhy{}}\PYG{o}{\PYGZhy{}}\PYG{o}{\PYGZgt{}}
\PYG{p}{[} \PYG{p}{]} \PYG{n}{C}\PYG{o}{+}\PYG{o}{+} \PYG{n}{support} \PYG{k}{for} \PYG{n}{the} \PYG{n}{application}  \PYG{o}{\PYGZhy{}}\PYG{o}{\PYGZhy{}}\PYG{o}{\PYGZhy{}}\PYG{o}{\PYGZhy{}}
    \PYG{n}{System} \PYG{n}{Monitoring} \PYG{n}{Options}  \PYG{o}{\PYGZhy{}}\PYG{o}{\PYGZhy{}}\PYG{o}{\PYGZhy{}}\PYG{o}{\PYGZgt{}}
    \PYG{n}{Debugging} \PYG{n}{Options}  \PYG{o}{\PYGZhy{}}\PYG{o}{\PYGZhy{}}\PYG{o}{\PYGZhy{}}\PYG{o}{\PYGZgt{}}
\PYG{p}{[} \PYG{p}{]} \PYG{n}{Disk} \PYG{n}{Interface}  \PYG{o}{\PYGZhy{}}\PYG{o}{\PYGZhy{}}\PYG{o}{\PYGZhy{}}\PYG{o}{\PYGZhy{}}
    \PYG{n}{File} \PYG{n}{Systems}  \PYG{o}{\PYGZhy{}}\PYG{o}{\PYGZhy{}}\PYG{o}{\PYGZhy{}}\PYG{o}{\PYGZgt{}}
\PYG{o}{\PYGZhy{}}\PYG{o}{*}\PYG{o}{\PYGZhy{}} \PYG{n}{Logging}  \PYG{o}{\PYGZhy{}}\PYG{o}{\PYGZhy{}}\PYG{o}{\PYGZhy{}}\PYG{o}{\PYGZgt{}}
    \PYG{n}{Management}  \PYG{o}{\PYGZhy{}}\PYG{o}{\PYGZhy{}}\PYG{o}{\PYGZhy{}}\PYG{o}{\PYGZgt{}}
    \PYG{n}{Networking}  \PYG{o}{\PYGZhy{}}\PYG{o}{\PYGZhy{}}\PYG{o}{\PYGZhy{}}\PYG{o}{\PYGZgt{}}
\end{sphinxVerbatim}


\subsection{after saving :}
\label{\detokenize{menuconfig:after-saving}}
you will need to rebuild the image to include changes in config

\begin{sphinxVerbatim}[commandchars=\\\{\}]
\PYG{g+gp}{\PYGZdl{}} west build
\end{sphinxVerbatim}


\chapter{Hacking stuff}
\label{\detokenize{hacking/hacking:hacking-stuff}}\label{\detokenize{hacking/hacking:hacking}}\label{\detokenize{hacking/hacking::doc}}

\section{debugging   the ds\sphinxhyphen{}d6 smartwatch}
\label{\detokenize{hacking/debugging:debugging-the-ds-d6-smartwatch}}\label{\detokenize{hacking/debugging::doc}}
\begin{sphinxVerbatim}[commandchars=\\\{\}]
\PYG{n}{The} \PYG{n}{ds}\PYG{o}{\PYGZhy{}}\PYG{n}{D6} \PYG{n}{has} \PYG{n}{a} \PYG{n}{serial} \PYG{n}{port}\PYG{o}{.}
\end{sphinxVerbatim}

The blackmagicprobe can lauch a debugger : west debug \textendash{}runnerblackmagicprobe.
The probe has a serial port /dev/ttyACM1 (linux : minicom \sphinxhyphen{}b 115200 \sphinxhyphen{}D /dev/ttyACM1)

The Segger Jlink, which is kind of standard in zephyr can use the the swd\sphinxhyphen{}connector for debug messages as well.

\begin{sphinxadmonition}{note}{Note:}
The Segger JLink offers the JLinkRTTViewer.
In order to use it, you can set this in ds\_d6\_defconfig (boardconfig)
CONFIG\_LOG=y
CONFIG\_USE\_SEGGER\_RTT=y
CONFIG\_SHELL=y
CONFIG\_SHELL\_BACKEND\_RTT=y
\end{sphinxadmonition}

sniffing memory

\begin{sphinxVerbatim}[commandchars=\\\{\}]
\PYG{g+go}{The JLink probe allowed to check memory at 0x5000504 and 0x50000514.}
\PYG{g+gp}{\PYGZsh{}}mem32 0x5000504 0x1
\PYG{g+go}{or write a value}
\PYG{g+gp}{\PYGZsh{}}w4 0x50000504 0x12345678
\PYG{g+go}{This allowed me to check GPIO ports.}
\end{sphinxVerbatim}


\section{howto generate pdf documents}
\label{\detokenize{hacking/latexpdf:howto-generate-pdf-documents}}\label{\detokenize{hacking/latexpdf::doc}}
sphinx cannot generate pdf directly, and needs latex

\begin{sphinxVerbatim}[commandchars=\\\{\}]
\PYG{n}{apt}\PYG{o}{\PYGZhy{}}\PYG{n}{get} \PYG{n}{install} \PYG{n}{latexmk}
\PYG{n}{apt}\PYG{o}{\PYGZhy{}}\PYG{n}{get} \PYG{n}{install} \PYG{n}{texlive}\PYG{o}{\PYGZhy{}}\PYG{n}{fonts}\PYG{o}{\PYGZhy{}}\PYG{n}{recommended}
\PYG{n}{apt}\PYG{o}{\PYGZhy{}}\PYG{n}{get} \PYG{n}{install} \PYG{n}{xzdec}
\PYG{n}{apt}\PYG{o}{\PYGZhy{}}\PYG{n}{get} \PYG{n}{install} \PYG{n}{cmap}
\PYG{n}{apt}\PYG{o}{\PYGZhy{}}\PYG{n}{get} \PYG{n}{install} \PYG{n}{texlive}\PYG{o}{\PYGZhy{}}\PYG{n}{latex}\PYG{o}{\PYGZhy{}}\PYG{n}{recommended}
\PYG{n}{apt}\PYG{o}{\PYGZhy{}}\PYG{n}{get} \PYG{n}{install} \PYG{n}{texlive}\PYG{o}{\PYGZhy{}}\PYG{n}{latex}\PYG{o}{\PYGZhy{}}\PYG{n}{extra}
\end{sphinxVerbatim}


\chapter{About}
\label{\detokenize{about:about}}\label{\detokenize{about::doc}}
The Desay D6 smart watch, contains the nordic nrf52832 microcontroller.

This watch has an serial port.

In fact it is a small computer on your wrist, with a battery and screen, and capable of bluetooth 4+ wireless communication.

\begin{sphinxVerbatim}[commandchars=\\\{\}]
\PYG{n}{A} \PYG{n}{word} \PYG{n}{of} \PYG{n}{warning}\PYG{p}{:} \PYG{n}{this} \PYG{o+ow}{is} \PYG{n}{work} \PYG{o+ow}{in} \PYG{n}{progress}\PYG{o}{.}
\PYG{n}{You}\PYG{l+s+s1}{\PYGZsq{}}\PYG{l+s+s1}{re likely to have a better skillset then me.}
\PYG{n}{You} \PYG{n}{are} \PYG{n}{invited} \PYG{n}{to} \PYG{n}{add} \PYG{n}{the} \PYG{n}{missing} \PYG{n}{pieces} \PYG{o+ow}{and} \PYG{n}{to} \PYG{n}{improve} \PYG{n}{what}\PYG{l+s+s1}{\PYGZsq{}}\PYG{l+s+s1}{s already there.}
\end{sphinxVerbatim}


\section{What is already working :}
\label{\detokenize{about:what-is-already-working}}\begin{itemize}
\item {} 
wireless hack : one can install espruino wirelessly (github.com/fanoush/ds\sphinxhyphen{}d6)

\item {} 
zephyr RTOS with working screen

\end{itemize}



\renewcommand{\indexname}{Index}
\printindex
\end{document}